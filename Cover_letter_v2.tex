\makeatletter
\def\input@path{{rho-class/}}
\makeatother
\documentclass[coverletter, 11pt]{rho}
\usepackage{ragged2e}
\usepackage[none]{hyphenat}
% override Rho margins (geometry already loaded by the class)
\geometry{a4paper, margin=2.5cm}

\begin{document}
\RaggedRight

\coverdate{23 December 2025}

\begin{flushleft}
\textbf{Prof. Sara Goodacre} \\
Editor-in-Chief \\
\textit{Heredity} \\
Email: heredity@nature.com \\
\end{flushleft}

\begin{coverletter}
Dear Prof. Goodacre,

We are pleased to submit our manuscript, \textbf{``Genomic Evidence for Altitude-Driven Adaptive Divergence in a Color-Polymorphic Montane Insect''}, for consideration in \textit{Heredity}. This study investigates how environmental gradients shape genomic and phenotypic variation by integrating population genomics, genotype--environment associations, and phenotypic analyses in a non-model montane insect endemic to the Fırtına Valley, T\"{u}rkiye.

Our results demonstrate that steep altitudinal variation drives adaptive genetic divergence despite ongoing gene flow. Using RAD-sequencing data from 71 individuals sampled across elevations from 450 to 2,300 meters, we identified hundreds of putatively adaptive SNPs, several of which showed strong associations with both altitude and a striking color polymorphism. Bidirectional allele frequency shifts at key loci suggest divergent selection maintains alternative alleles across environments.

We believe this work aligns well with \textit{Heredity}'s readership due to its emphasis on:

\begin{itemize}[leftmargin=*,nosep]
  \item The genetic architecture of local adaptation in natural populations
  \item Genotype--environment associations and population genetic structure
  \item The application of genomic tools to dissect adaptive phenotypic variation in a non-model system
\end{itemize}

More broadly, our findings contribute to ongoing debates on how environmental heterogeneity structures genetic diversity and how adaptive signatures persist under high connectivity. The study also offers a transferable analytical framework for exploring adaptation in other non-model taxa.

This manuscript is original, has not been published or submitted elsewhere, and all authors have approved its submission. We have no conflicts of interest to declare.

Thank you for considering our work for publication in \textit{Heredity}. We would be delighted by the opportunity to contribute to your journal.

\end{coverletter}

\coversignature{
Sincerely,\\
\vspace{0.2cm}
\includegraphics[width=4cm]{signature.png}\\
İsmail K. Sağlam (corresponding author)\\
on behalf of all co-authors\\
Department of Molecular Biology and Genetics,\\
Koç University, İstanbul, Türkiye\\
\href{mailto:iksaglam@ku.edu.tr}{iksaglam@ku.edu.tr}\\
\href{https://orcid.org/0000-0003-3136-7334}{ORCID: 0000-0003-3136-7334}
}

\end{document}